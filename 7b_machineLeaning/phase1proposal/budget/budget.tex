%% {\em Complete the Research and Related Budget form in accordance with
%%   the instructions on the form (Activate Help Mode to see
%%   instructions) and the following instructions.  You must complete a
%%   separate budget for the support requested (one budget for the
%%   applicant and a budget for each subawardee).  The form will generate
%%   a cumulative budget for the total project period.  You must complete
%%   all the mandatory information on the form before the NEXT PERIOD
%%   button is activated.  You may request funds under any of the
%%   categories (other than ``Participant/Trainee Support Costs'') listed
%%   as long as the item and amount are necessary to perform the proposed
%%   work, meet all the criteria for allowability under the applicable
%%   Federal cost principles, and are not prohibited by the funding
%%   restrictions in this announcement (See PART IV, G).}
 
%\section{Budget Justification} %(Field K on the form). 

%% {\em Provide the required supporting information for all proposed
%%   costs (See R\&R Budget instructions.  Attach a single budget
%%   justification file (will not count in page limit) for the entire
%%   project period in Field K.  The file automatically carries over to
%%   each budget year.  Please note, if you are selected for an award,
%%   additional budget information will most likely be required.

%%   Notes regarding Budget:
  
%%   \begin{itemize}
%%     \item Although there is no absolute cap on indirect costs, grant
%%       applications will be evaluated for overall economy and value to
%%       DOE.

%%     \item Tuition expenses are only allowable if requested from a
%%       subcontractor that is a University and if the amount requested
%%       for tuition is reasonable and comparable to the amount a student
%%       would be paid for performing research during the grant period.

%%     \item Travel funds must be justified and directly related to the
%%       needs of the project.  Travel expenses for technical conferences
%%       are not permitted unless the purpose of attending the conference
%%       directly relates to the project (e.g., to present research
%%       results of the project).  Funds to cover travel expenses outside
%%       of the U.S. is considered an unallowable cost unless written
%%       approval has been obtained from the SBIR/STTR Program Manager.

%%      \item Grants may include a profit or fee for the small business.

%%      \item Any commercial and/or in-kind contribution to the project
%%        should be reflected in the project narrative and not included
%%        on the budget pages.
%% \end{itemize}
%% }

\section{Section A -- Senior/Key Personnel}
Dr. Ben O'Neill is the PI and will lead the technical effort with
assistance from Dr. Gerald Sabin. 

\section{Section B -- Other Personnel}
None.

\section{Section C -- Equipment Description (Items Exceeding \$5,000)}
No major equipment.

\section{Section D -- Travel}
There will be two meetings with the DOE Program Manager and interested
DOE scientists. The first will be a kick-off meeting (i.e., the DOE PI
Meeting) to start the contract and ensure that the DOE needs are
met. The second is a review meeting toward the end of the project to
discuss the project status.  RNET is budgeting a minimal amount for
travel. Any additional RNET travel expenses (e.g., if both meetings
are in person) will be paid from internal funding.

\section{Section E -- Participant/Trainee Support Costs}
None
  
\section{Section F -- Other Direct Costs}
The University of Oregon is a subcontractor on this project. The UO budget
and a letter of commitment are attached to this proposal submission. Dr. Boyana Norris will lead the technical effort at UO. 

\section{Section H -- Indirect Costs}
RNET uses a labor overhead of 51\%. This overhead includes, fringe
benefits (holidays, sick leave, vacation, health-care benefits and
retirement contributions), payroll taxes, and other indirect charges.

RNET charges 7\% for ``General and Administrative'' tasks (G\&A),
including accounting, payroll services, office rent, office supplies,
telephone, Internet connection, etc.

\section{Section J -- Fee}
RNET charges a fee (profit) of 7\%.  This fee is consistent with other
accepted RNET proposals.



