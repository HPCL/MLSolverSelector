\emph{RNET estimates sales revenue of \$5M and licensing revenue of \$0M over the first 10 years of commercialization.}

\section{Market Opportunity}


\subsection{ Product Description}



\subsection{Potential Customers and End-users} 



\subsection{Competition} 

\section{Intellectual Property}
To protect our Intellectual Property (IP), we will use “Trade Secrets”, which will allow RNET to protect the underlying technologies and process used for product development. Trade secrets can make copying some technologies more difficult. If a need for patents arises, we will employ the services of well-known patent law firms in Dayton, OH, and in Columbus, OH.

\section{Company/Team}
RNET Technologies (i.e., RNET), located in Dayton, Ohio, was founded in June 2003 as a "C" Corporation organized under the laws of the State of Delaware. Dr. V. Nagarajan (RNET President), who has an MBA in Marketing, will develop the marketing and the commercialization plan for the proposed Material and Computational Science workflow platform. He has over twenty years of executive and senior management experience. At RNET, he is leading the effort to commercialize the SmartNIC, the WattProf power monitoring board, HDTomoGPR product (a high resolution ground penetration product), the rad-hard ULP FPGA and the ROICs. RNET will hire a seasoned Sales Manager and it intends to hire such an individual, as needed, from industry to assist Dr. Nagarajan.
 Dr. Ben O�'Neill is the PI for this project. Dr O'Neills background is in HPC algorithms and parallel-time-integration. Dr. O’Neill is also the PI for the Vera-Workbench project, a project with the goal of developing a python based GUI for the VERA suite of tools, and is the lead developer for Cloudbench, a unique web-enabled workflow and provenance portal for open and government sourced HPC software. Dr. Gerald Sabin, Project Manager, has significant HPC and networking experience. Dr. Sabin has experience in the development of a web-based portal for nuclear engineering simulations (in a Phase II SBIR) and leading many other simulation and HPC based SBIR projects, including the development of MapReduce implementations (via leading the iNFORMER project at RNET).

\section{Revenue Forecast}
RNET estimates sales revenue of \$5million of sales over the first 10 years of commercialization will be realized through follow on contracts to develop additional features with the NWChem team. The Phase II commercialization proposal will include updated revenue figures with product price estimates and unit sales.
