\emph{RNET estimates sales revenue of \$5M and licensing revenue of \$15M over the first 10 years of commercialization.}

\section{Market Opportunity}
Computational chemistry codes provide vital simulation capabilities to a wide range of scientists. A DOE use case includes energy production and clean fuel technology; through the simulation of kinetics and dynamics of chemical processes. Drug discovery is one example of an important commercial application area. To discover and bring a new drug to market takes an average of 12 years and \$12B \footnote{\url{https://insidehpc.com/2017/10/accelerating-quantum-chemistry-drug-discovery/}}. Improving chemical simulations can improve the drug candidates, reducing the testing required and improving the drug quality. 

These computational chemistry simulation tools require vast computational resources. Therefore, performance is a critical component to the successful marketing (and application) of simulation tools. Therefore, performance and scalability improvements are required in the existing codes. One path that has the potential to provide massive improvements, are automatic optimization tools to reduce the operation count and runtime requirements of tensor codes. Fortunately, it is possible to develop a complier to perform tensor contraction to output optimized tensor codes. 

The proposed OpMin tool will be part of the computational chemistry software market, which is experiencing a rapid growth and provides a fertile basis to introduce improved products. The computational chemistry software market is expected to have a CAGR of 11\% through 2022\footnote{\url{https://www.tullahomanews.com/news/business/global-chemical-software-market---chemical-process-simulation-segment/article_efff78b3-eee9-5ab9-a6f6-8c8dbb1f7614.html}}. 

The NWChem software suite from Pacific Northwest National Laboratory (PNNL) is among the most scalable and widely used quantum chemistry software in computational chemistry. It has been selected as a DOE Exascale computing Program (ECP) application. We will collaborate with them to ensure that the OpMin tools has the features and usability necessary for inclusion in production computational chemistry codes.

\subsection{Product Description}
The Tensor Contraction Tool will provide computational scientists and code developers a simple mechanism to include automatic and semi-automatic support to allow end users to write simple applications (in a domain specific tensor language), and generate high performance scientific code that is optimized for existing High Performance Computing Architectures as well as Exascale Computing Architectures. 

These tools will allow the end user to develop high performance codes in hours or days, that would require months without such a tool.

Initial integrate will be performed in NWChem, to demonstrate the feature set and performance benefits of OpMin. 

\subsection{Potential Customers and End-users} 
End users include computational chemistry scientist in industry and academia. These users produces chemical components and/or chemical products. Example application areas include drug discovery, specialty chemicals, fuel/petroleum products, organic materials, polymers, consumer products. Just a few of the many example companies include Bayer AG, Dow Chemical, Lubrizol, Unilever, RIKEN, DuPont, BP, and BASF.\footnote{http://www.dtic.mil/dtic/tr/fulltext/u2/a467500.pdf} 

Customers are those developing computational chemistry codes, who would license the technology to apply OpMin in their codes. Examples include QChem, Gaussian, GAMESS, MOLPRO, PWmat, TeraChem, ADF, and NWChem.

\subsection{Competition} 
Competitors include research groups and companies with the ability to develop high performance domain specific languages, and groups developing computational chemistry codes. Developing such a software requires a solid understanding of the chemistry that leads to the design choices and optimizations opportunities in computational chemistry codes, coupled with the ability to understand the bottlenecks and challenges in developing high performance and Exascale codes.

Therefore, the competition will largely be research and commercial groups developing computational chemistry codes; this group represents competitors and customers. They have the potential ability to develop the domain language and compiler, or something functional similar. However, we believe it will be cost conscious to license the Tensor Contraction Tool and further we believe that we are in a unique position to develop tools that create high performance codes. 

Example competitors include; QChem, Gaussian, GAMESS, MOLPRO, PWmat, TeraChem, ADF, the NWChem team at PNNL.

\section{Intellectual Property}
To protect our Intellectual Property (IP), we will use "Trade Secrets"��, which will allow RNET to protect the underlying technologies and process used for product development. Trade secrets can make copying some technologies more difficult. If a need for patents arises, we will employ the services of well-known patent law firms in Dayton, OH, and in Columbus, OH.

It is expected that details of the implementation will be released via scientific journals, but the source code of the implementation will remain with RNET. 

\section{Company/Team}
RNET Technologies (i.e., RNET), located in Dayton, Ohio, was founded in June 2003 as a "C" Corporation organized under the laws of the State of Delaware. Dr. V. Nagarajan (RNET President), who has an MBA in Marketing, will develop the marketing and the commercialization plan for the proposed Material and Computational Science workflow platform. He has over twenty years of executive and senior management experience. At RNET, he is leading the effort to commercialize the SmartNIC, the WattProf power monitoring board, HDTomoGPR product (a high resolution ground penetration product), the rad-hard ULP FPGA and the ROICs. RNET will hire a seasoned Sales Manager and it intends to hire such an individual, as needed, from industry to assist Dr. Nagarajan.

 Dr. Gerald Sabin, Principal Investigator, has significant HPC experience. Dr. Sabin has experience in the development of optimized linear algebra kernels for HPC applications (in a Phase II SBIR) and leading many other simulation and HPC based SBIR projects, including the development of MapReduce implementations (via leading the iNFORMER project at RNET).

Dr. Ben O'Neill is Senior Research Scientist at RNET. Dr O'Neills background is in HPC algorithms and parallel-time-integration. Dr. O'Neill is also the PI for the Vera-Workbench project, a project with the goal of developing a python based GUI for the VERA suite of tools, and is the lead developer for Cloudbench, a unique web-enabled workflow and provenance portal for open and government sourced HPC software. 

Dr. Sadayppan (Ohio State University Professor) is included as part of the OSU subcontract, and will be a key advisor for the development and commercial opportunities of the Tensor Contract Tool. Dr. Sadayappan has extensive knowledge of computational chemistry codes, applications, and users. He led the development of the original Tensor Contraction Engine developed in close coordination with the PNLL NWChem team, and a wide range of computational chemists. Dr. Sadayappan will help ensure that the Tensor Contraction Tool has the features require for high performance and usability. 

\section{Revenue Forecast}
RNET estimates sales revenue of \$2 million of sales over the first 10 years of commercialization. Direct sales will be realized through follow on contracts to develop additional features with the NWChem team and to apply the OpMin tools to specific end user application codes. RNET estimates licensing revenues of \$10 million over the first 10 years of commercialization. Licensing revenues will be realized through collaboration with customers who will integrate the operation minimization tools into their computational chemistry codes, and charging a percentage based fee on sales that include the Tensor Contraction Tools and OpMin.

The Phase II commercialization proposal will include updated revenue figures with more specifics on the licensing model and fee.
