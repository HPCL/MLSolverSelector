%Company Commercialization History. 
%{\em If the answer to the this question is yes, provide your
%  commercialization history in a table which includes: title of the
%  project; source of funding (if Federal, indicate whether SBIR, STTR
%  or other); year the funding was received; total sales of the
%  resulting product or service (include sales by your company and any
%  licensee- identify the licensee); and total revenues obtained from
%  commercialization (identify sources of these revenues).  Attach in
%  the block provided.}

RNET Technologies (i.e., RNET), located in Dayton, Ohio, was founded
in June 2003 as a ``C'' Corporation organized under the laws of the
State of Delaware. The overall mission of RNET is to develop
leading-edge software products that will meet the needs
of the DOE, DOD, NASA, Prime Contractors and other Commercial
Companies. To meet this mission, RNET is pursuing R\&D and Product
Development in  HPC (high performance computing).

RNET received its first set of two Phase I SBIR contracts in 2004. As
of November 2018, RNET has received 18 Phase II SBIR/STTR contracts of
which three are on-going while the remaining  have been
completed. Five of the Phase II SBIR projects have been funded by the
Air Force, three have been funded by NASA, one by DARPA, and the
remaining nine have been funded by the DOE.

As of October 2018, RNET has received more than 40 Phase I projects
from the Air Force, Navy, MDA, NASA, NSF, DARPA, SOCOM, and DOE.

In addition to the SBIR/STTR projects, RNET has also supported a
non-SBIR BAA-type support contract (F8650-10-D-1750) named CIRE
(Center for Innovative Radar Engineering), which focuses on ``next
generation'' GHz and THz frequency radars. Moreover, RNET has received
a Phase III contract (\$100K) from Sandia National Laboratories, on WattProf HPC 
power monitoring product. 

As a company, we have strong expertise to develop advanced software for High 
Performance Computing applications in several domains.

The first Phase II SBIR completed by RNET was funded by the Air Force
(FA8650-05-C-4303). It developed an innovative ``virtual objects based
compression (VOBC)'' video compression software package. We received
Phase II Enhancement funding of a \$100K that included \$50K funding
from DARPA to enhance the capabilities of the VOBC software and to
investigate its use in UAVs. We were able to generate \$29K revenue by
conducting video processing work for a small company that was
developing an innovating 3-D video technology. In addition, we also
generated \$337K from an Air Force funded video/image processing system
for a mobile IED detection system that also used Android
platform. This Android work led to generation of \$4K from a research
group at the Ohio State University.

The second Phase II SBIR we completed, which was also funded by the
Air Force (FA8650-06-C-1019), developed 2 GHz/pixel sampling ROIC
(read-out integrated circuit). Based on this expertise, we were able
to work with a small company named Traycer, which was developing a
ROIC/Sensor as part of a Tera-Hertz camera system. We designed and
built off-chip ROIC hardware including a high-speed, low-noise signal
amplifier and digitizer board, and this project generated \$20K in
revenues. 

The third Phase II SBIR we completed was funded by the DOE (Grant
DE-FG02-05-ER84163). In this project, we designed and built a 10
Giga-bit Ethernet SmartNIC. This NIC is based on the OCTEON-Plus 12
core processor from Cavium Networks. It also has 2 giga-bytes of
on-board memory. The innovation is offloading and we needed to
develop a range of network-level and application-level offload engines
(and the required drivers and firmware) including encryption and SSL
offload engines for Globus/GridFTP, and these were accomplished in the
next two STTR/SBIR projects (Grants DE-FG02-08-ER86360 and
DE-SC-0002182) that were completed. 
%Currently, we are extensively
%testing all the software modules for performance. 
We have also
invested about \$60K in building about 25 SmartNICs. Our plan is to
loan these SmartNICs to potential customers in the DOE,
universities, and the Industry. We have also invested an additional
\$70K to develop a series of 10, 20 and 40 giga-bit SmartNICs based on
the next-generation OCTEON-II processor which has 32 cores running at
almost 1.5 GHz. All the offload engines that we have developed for the
OCTEON-Plus processor based 10 GigE SmartNIC will also be easily
portable to the OCTEON-II based SmartNICs.
%, and we are planning to sell
%these in 2017 and subsequent years. The plan is to generate
%\$50K revenue in 2018, and about \$250 to \$400K in the subsequent 5 to
%10 years.

In another  completed DOE Phase II SBIR project
(DE-SC-0002434) we have optimized the PETSc (portable extensible
toolkit for scientific computing) library for emerging architectures
including multi-core processors and GPUs. As part of that project, the
team investigated algorithms, data structures, and techniques to
enable applications based on the PETSc library. PETSc is an open
source product, and thus commercial sales of the optimized PETSc
library are not feasible.  RNET is  pursuing non-SBIR contracts
from the Air Force as part of commercialization.

RNET has also developed hardware and software tools as part of DOE
STTR Grant DE-SC-0004510 for fine-grain monitoring of power
consumption when applications run on compute nodes in a HPC
cluster. The first release of this product is ready and several evaluation
toolkits have been distributed to potential customers. 
%Beside the Phase III contract from Sandia 
%national labs, we will continue the commercialization of the product in the coming years.
%We anticipate that we can generate \$200 to \$300K per year over the next  5 to 10 years.

On an Air Force funded Phase II STTR (FA9550-12-C-0028)) we
have made optimizations to the Kestrel, an Air Force CFD and CSD
simulation code. These optimizations have been included in the
production Kestrel release. In addition, RNET is pursuing additional
opportunities for Phase III funding to develop additional
optimizations.

%We are also developing highly-innovative radiation-hardened FPGAs for
%space applications for NASA as part of two Grants NNX-10-C-B47C
%(recently ended in 2015) and NNX-12-C-A84C (ending in early 2017). We
%are continuously interacting with Boeing-Space Division in California,
%who have expressed interest in these for satellite applications. We anticipate that it 
%will take about another 12 months for the products to be
%space-qualified, and hence we will not be able to generate sales revenues until
%early 2018. 

We are still in the early commercial stages of sveral other Phase II
projects. As part of a NASA Phase II (NNX14CG06C) we are developing
FPGA optimization tools using empirical tuning. We are pursuing two
commercialization routes. The first is licensing through a major FPGA
vendor such as Xilinx, and the second is Phase III NASA funding to
apply to tool to NASA FPGA designs. An improved GUI was recently developed for this project, and released in August of 2018. As part of a DOE Phase II project
(DE-SC0011322) we have developed HD TomoGPR, a novel below ground
imagining system for fine root analysis using tomography and Ground
Penetrating Radar. This system will be sold to domain scientists and
marketed to other market segments (e.g., concrete and bridge health
analysis). We recently completed an initial experiment in cooperation
with DuPont/Pioneer and are pursuing a non-SBIR project to adapt the
product for runway analysis. The initial prototype is being delivered to PNNL. As part of another DOE Phase II project
(DE-SC0011312) we have developed optimizations to BigData
platforms. Joint sales opportunities are being explored with major Big
Data players including Lexis Nexis and Hortonworks. We have 
been award the Yarn Ready certification for a Map Reduce like API. These tools will be released on RNET's website. Finally, we
are just beginning a Phase II project developing an automated linear
solver selector to improve the runtime, accuracy, stability, and power
of scientific simulations. This tool will be integrated into
government applications and commercialize as a tool to be integrated
into third-party commercial applications at companies such as ANSYS. Finally, we have recently commenced a DOE Phase II to develop CloudBench:NE, a web based simulation, provenance, and sharing workbench for nuclear engineering using NEAMS tools.
