\section{Anticipated Public Benefits}
%% {\em Discuss the technical, economic, social, and other benefits to the
%%  public as a whole anticipated if the project is successful and is
%%  carried over into Phases II and III.  Identify specific groups in the
%%  commercial sector as well as the Federal Government that would
%%  benefit from the projected results.  Describe the resultant product
%%  or process, the likelihood that it could lead to a marketable
%%  product, and the significance of the market.}

The proposed project will facilitate runtime performance tuning of numerical simulations based on 
both the computational model and the compuation architecture, in terms of any combination of measurable metrics ( 
e.g., CPU time, Memory consumption, energy efficiency). Numerical simulations, widely applicable across various scientific 
disciplines, are a key component in the development of most, if not all, of the cutting edge technologies being developed 
by the DOE and other U.S government agencies. By automating the process of tuning simulations for a specific architecture, 
SASI will provide the the computational scientists, numerical software developers, and 
electrical engineers with a mechanism for letting the software decide the best performing 
method without the need for labor-intensive experimentation, and thus focus on more 
scientific aspects of their applications. Moreover, by ensuring applications are optimized for the architecture
for which they are being used on, SASI will inform the efficient usage of the nations current and emerging computational 
resources. The targeted customers include nuclear power companies, DoD and its Prime Contractors, 
NASA divisions, DOE agencies, CFD software providers, oil and gas companies, semiconductor 
design companies etc.











