\section{Facilities/Equipment}
%{\em Describe available equipment and physical facilities necessary to
%  carry out the Phase I effort.  Items of equipment to be leased or
%  purchased must be described and justified in this section.
%  Equipment is defined as an article of tangible, non-expendable,
%  personal property, including exempt property, charged directly to
%  the award, having a useful life of more than one year, and an
%  acquisition cost of \$5,000 or more per unit. Title to equipment
%  purchased under this award lies with the government.  It may be
%  transferred to the grantee where such transfer would be more cost
%  effective than recovery of the property by the government.  Awardees
%  wishing to obtain title should contact their Contract Specialist
%  prior to project completion for the procedure to follow.
%
%  If the equipment, instrumentation, and facilities are not the
%  property of the applicant and are not to be purchased or leased, the
%  source must be identified and their availability and expected costs
%  specifically confirmed in this section.  A principal of the
%  organization that owns or operates the facilities/equipment must
%  certify regarding the availability and cost of facilities/equipment
%  and any associated technician cost; a copy of this certification
%  must be submitted as part of the grant application.
%
%  To the extent possible in keeping with the overall purposes of the
%  program, only American-made equipment and products should be
%  purchased with financial assistance provided under both Phase I and
%  Phase II awards.}

%\subsection{RNET Facilities}



RNET currently has 9 development computers and a 10-node development cluster 
that can be used for development and testing in this effort. Each cluster node 
has two quad-core or hexa-core XEON CPUs, 24-32GB of DRAM, 500+GB of local disk. 
Two data networks are available, a COTS 1 Gbps Ethernet network and a 10 Gbps 
Ethernet network. The University of Oregon currently has several clusters connected with high performance networks (such as InfiniBand and 10GigE).
The facilities in the CIS Department at University of Oregon including the High Performance Computing infrastructure will be available 
for this research. 

The primary goal of this Phase I project is to develop the framework for developing and using the proposed machine learning based performance tuning models. 
To that end, RNET and UO has the tools (software and hardware) to evaluate and develop the technologies proposed as part of this Phase I project. However, it is important to note  that large-scale testing on a variety of compute resources with a range of core counts, networks and hardware will be a key and important component of the Phase II proposal. The requests for the required resource allocations will be made as part of the Phase II application, but are mentioned here to indicate our desire to use ASCR resources should a Phase II award be granted.  
