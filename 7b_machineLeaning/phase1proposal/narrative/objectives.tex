\section{Technical Objectives}
%% {\em State the specific technical objectives of the Phase I effort,
%%   including the questions it will try to answer to determine the
%%   feasibility of the proposed approach.}
 
To demonstrate the technical feasibility of the proposed approach, the Phase I effort will focus on developing a functional proof of concept for the 
SASI framework. The specific objectives RNET Technologies, and its subcontractor (UO) will pursue during the Phase I project are:

\begin{itemize}
  \item Develop the core interfaces and software indirections required to create a framework that 
  guides the user through the process of building and using an accurate and reliable SASI model.
  \item Prototype a range of tools for ensuring the SASI models are valid, including automated testing and analysis of the features, training data and model. 
  \item Investigate of approaches for reusing, modifying and/or re-purposing training data obtained on other architectures and/or from alternative algorithms or application areas in new models and for new architectures. 
  \item Demonstrate the effectiveness of the SASI framework by applying it to graph algorithms, a numerical subproblem that
  project team has experience developing SASI models for, but for which no sophisticated integration framework has been developed. 
\end{itemize}

By achieving these four objectives, RNET will demonstrate the scientific and commercial potential of the proposed package. The investigation into approaches for re-using data as well as the development of automated tools for assessing the validity of the model and data represent the key and novel scientific contributions of the proposed project. While we will be able to reuse some components of the SolverSelector API ( i.e., the Sqlite3 database interface ), the development of the core framework  in itself is a difficult computer science problem, primarily because of the requirement to support generic algorithms. 

The Phase II effort will include investigating efficient data collection algorithms and implementations such as a fork-join model to provide fast in-line data collection that efficiently utilizes the available computational resources and an investigation into ranking based machine learning models. Large scale testing of the models will also occur in Phase II. 
