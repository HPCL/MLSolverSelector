\section{Technical Objectives}
%% {\em State the specific technical objectives of the Phase I effort,
%%   including the questions it will try to answer to determine the
%%   feasibility of the proposed approach.}
 
To demonstrate the feasibility of the proposed approach,  we will develop a proof of concept for the SASI framework as introduced in the previous section. 

The requirements being addressed are leveraging the project teams extensive research into automatic solver selection
for linear solvers to develop a machine learning framework for automatic performance tuning of both generic numerical algorithms 
and simulations that is applicable in a broad range application areas. In order to demonstrate feasibility of the proposed approach in Phase I, RNET Technologies, and its subcontractor (UO) will pursue the following objectives.

\begin{itemize}
  \item Generalize and, where-ever possible, automate, the SolverSelector API to create a SASI framework that 
  guides the user through the process of building and using an accurate and reliable SASI model, including a robust set of machine learning analysis tools that test, analyze and optimize the feature set, data set, and model based on a user specified set of performance metrics. 
   
  \item Investigation of approaches for reusing, modifying and/or re-purposing training data obtained on other architectures and/or from alternative algorithms or application areas in new models and for new architectures. This will include an investigation into implementation and architecture specific features such as the processor speed, available memory and memory access patterns to provide informed predictions when switching to an architecture for which limited training data is available. 
  
  \item Demonstrate the effectiveness of the SASI framework by applying it to graph algorithms, a numerical subproblem that
  project team has experience developing SASI models, but for which no sophisticated integration framework has been developed. 
  
\end{itemize}

By achieving these three objectives, RNET will demonstrate the scientific and commercial potential of the proposed package. In particular, the investigation into architecture and implementation features will allow us to demonstrate the capability of SASI models to automatically adapt to not only changes in the computational model, but also to changes in the compute architecture. Likewise, the demonstration of SASI applied to graph algorithms, an area where the project team has prior experience developing SASI models, but for which model building and integration was a manual process, will all the project team to fully highlight the simplicity of the SASI framework, as well as the performance gains that can be achieved using SASI models. 

The Phase II will be primarily focused on further generalizing and improving the generic machine learning tools proposed in this proposal, including full scale development and testing of the tools developed during Phase I.  This will include investigating efficient data collection algorithms and implementations such as a fork-join model to provide fast in-line data collection that efficiently utilizes the available computational resources and an investigation into ranking based machine learning models. Mechanisms and software indirections required for utilizing and managing automatic algorithm selection using implementations from separate and distinct solver packages will also be developed as part of the Phase II effort. 

