\section{Technical Objectives}
%% {\em State the specific technical objectives of the Phase I effort,
%%   including the questions it will try to answer to determine the
%%   feasibility of the proposed approach.}
 
To demonstrate the feasibility of the proposed approach,  we will develop a proof of concept for the SASI framework as introduced in the previous section. 

The requirements being addressed are leveraging the project teams extensive research into automatic solver selection
for linear solvers to develop a machine learning framework for automatic performance tuning of both generic numerical algorithms 
and simulations that is applicable in a broad range application areas. In order to demonstrate feasibility of the proposed approach in Phase I, RNET Technologies, and its subcontractor (UO) will pursue the following objectives.

\begin{itemize}
  \item Generalize and, where-ever possible, automate, the SolverSelector API to create a SASI framework that 
  guides the user through the process of building and using an accurate and reliable SASI model.  Develop a robust set of machine learning analysis tools that test, analyze and optimize the feature set, data set, and model based on a user specified set of performance metrics. 
  
  \item Investigate efficient data collection algorithms and implementation such as a fork-join model to inform fast, efficient and
  simple in-line data collection that efficiently utilize the available computational resources. Investigate approaches for reusing, modifying and/or re-purposing training data obtained on other architectures and/or from alternative algorithms or application areas in new models and for new architectures.
  
  \item Develop mechanisms and software indirections for utilizing and managing automatic algorithm selection using implementations from separate and distinct solver packages. Demonstrate the effectiveness of such approaches by using the framework to begin developing a model for a new numerical subproblem classification systems based ODE solvers from packages like PETSc, libMesh and Trilinos.  
 

\end{itemize}

By achieving these three objectives, RNET will demonstrate the scientific and commercial potential of the proposed package. In particular, the Phase I product will provide a through proof of concept for the SASI framework. In particular, the development of algorithms for efficient data collection and reuse across domains, algorithms and architectures will address the primary concern associated with SASI models, i.e., the cost of building the training data sets. Likewise, the time spent developing a mechanism for managing and utilizing algorithms from a different numerical libraries will highlight the fact that SASI is useful even in cases where an extensive, uniform and production grade algorithm such as the linear solver interface in PETSc does not exist. 
The Phase II will be primarily focused on further generalizing and improving the generic machine learning tools proposed in this proposal, including full scale development and testing of the work developed during Phase I. The Phase II effort will also include an investigation into ranking based machine learning models as discussed earlier. 

