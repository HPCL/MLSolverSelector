\section{Principal Investigator and other Key Personnel}
%{\em The Principal Investigator (PI) must be knowledgeable in all
%  technical aspects of the grant application and be capable of leading
%  the research effort and meet the requirements described in Part III,
%  C.  Describe the effort to be performed by the PI during the
%  project.}

\subsection{Ben O'Neill}
Ben O'Neill is a Research Scientist at RNET and will be the PI for this project. Ben is a full time employee at
RNET and has sufficient time to dedicate to this project. Dr. O'Neill is a Permanent Resident of the United States 
and a Citizen of New Zealand. Dr. O'Neill, in collaboration with Dr. Boyana Norris, is currently leading the research effort for the  
ongoing Phase II DOE project (DE-FOA-001490) for the Automated Solver Selection for Nuclear Engineering 
Simulations described in section~\ref{sec:linearsolvers}. Dr. O'Neill is also the PI for the ongoing Phase I DOE project for 
the development of verification and validation toolkit for large-scale numerical simulation. Other projects Dr. O'Neill
has been involved in include the Phase I DOE Vera workbench project and as the lead developer in the ongoing Phase II 
DOE project for the development of Cloudbench, a web-enabled interface for remote execution and visualization for nuclear 
physics tools. His background is in Applied mathematics, high performance computing,
and parallel-time integration. His work includes a detailed investigation into parallel-time-integration
with MGRIT for nonlinear problems, an enhanced MGRIT algorithm based on Richardson extrapolation and he is also
involved in implementing several features currently under development as part of the parallel in time XBraid project.

\subsection{Dr. Gerald Sabin, Principal Investigator}
Dr. Gerald Sabin, senior researcher at RNET and is a US Citizen.  Dr.  Sabin is a full time employee of RNET, and has
sufficient time to dedicate to project tasks as indicated in the cost
proposal. Currently, he is working on several Scientific Computing (HPC) 
SBIR/STTR projects at RNET. He is the PI the 
ongoing Phase II DARPA project developing a linear solver library for graph applications  and on the Phase II SBIR project 
(DE-SC0015748) developing a ``Web Infrastructure for Remote Modeling and
Simulation of Nuclear Reactors and Fuel Cycle Systems''. He is also the PI on the 
ongoing Phase II DOE project (DE-FOA-001490) for the Automated Solver Selection for Nuclear Engineering 
Simulations. He has also worked on distributed memory, GPU, multi-core and SIMD 
optimizations to the Air Force's Kestrel code (DOD Contract\#:FA9550-12-C-0028). He has also been the PI on several other
related projects including a NASA Phase I project developing SIMD
optimizations for Monte Carlo codes (NNX14CA44P), developing
parallelization optimizations for PETSc (DOE Contract\#:
DE-SC0002434), and developing data virtualization support and bitmap
indexing for massive Climate Modeling data sets (DOE Contract
\#:DE-SC0009520).



