\section{Work Plan} 
\label{sec:approach}
%% {\em Provide an explicit, detailed description of the Phase I research
%%   approach and work to be performed.  Indicate what will be done, by
%%   whom (small business, subcontractors, research institution, or
%%   consultants), where it will be done, and how the work will be
%%   carried out.  If applicant is making a commercial or in-kind
%%   contribution to the project, please describe in detail here.  The
%%   Phase I effort should attempt to determine the technical feasibility
%%   of the proposed concept which, if successful, would provide a firm
%%   basis for the Phase II grant application.

%%   Relate the work plan to the objectives of the proposed project.
%%   Discuss the methods planned to achieve each objective or task
%%   explicitly and in detail.  This section should be a substantial
%%   portion of the total grant application.} 


\section{Performance Schedule and Task Plan}
\label{sec:taskplan}

The goal of the Phase I effort will be to provide the reviewers with a clear idea 
of the scientific and commercial potential of ... The Phase I effort will primarily focus on .... The research and development topics described in Section \ref{sec:approach} 
will be addressed by the tasks described in the remainder of this section. 
Figure~\ref{fig:tasks} summarizes, at a high level, the dependencies among tasks 
and approximate anticipated task durations. The project duration is roughly 
divided into 1 month blocks. Specific details are included in the description 
of each task. 

RNET would like to present the project ideas and research plan to the DOE 
Program Manager and other interested scientists involved with NWChem.
This meeting will 
be scheduled soon after the Phase I contract is awarded. The Kickoff meeting 
will coincide with the Phase I SBIR PI meeting being hosted by the DOE. The 
meeting can be hosted at RNET, a DOE site suggested by the Program Manager or 
via a teleconference. RNET will submit a final report and present the report 
details along with a Phase II work plan to the DOE program manager and other 
interested scientists.

%\begin{figure}
%\centering
%\begin{tabular}{|c|c|c|c|c|c|c|c|c|c|}
%   \hline 
%   \multirow{2}{*}{ Time (Months): } & \multicolumn{9}{|c|}{ Phase I } \\
%   \cline{2-10} 
%   & 1 & 2 & 3 & 4 & 5 & 6 & 7 &8 & 9 \\
%   \hline 
%   \bcolumn{Task 1} & X & X & X & X & X  &   &   &   & \\
%   \hline
%   \bcolumn{Task 2} &   &  &  & X & X & X & X  & X  &  \\
%  \hline
%   \bcolumn{Task 3} &   &   &   &  &  & X & X &  X & X \\
%   \hline
%
%\end{tabular}
%\caption{Overview of task dependencies and time-line.}
%\label{fig:tasks}
%\end{figure}

\newcounter{taskCount}
\setcounter{taskCount}{0}

\refstepcounter{taskCount}\label{task:1}
\subsection{Task \ref{task:1}: }

\refstepcounter{taskCount}\label{task:2}
\subsection{Task \ref{task:2}: }


\refstepcounter{taskCount}\label{task:3}
\subsection{Task \ref{task:3}: }
