%% {\em The project summary must contain a summary of the
%%   proposed activity suitable for dissemination to the public.  This
%%   document must not include any proprietary or sensitive business
%%   information as the Department may make it available to the public.
%%   The project summary must not exceed 1 page when printed using
%%   standard 8.5'' by 11'' paper with 1'' margins (top, bottom, left and
%%   right) with font not smaller than 11 point.  Save this information
%%   in a file named Summary.pdf, and click on ``Add Optional Other
%%   Attachment'' to attach.
%%
%%  The summary must include: }
\section*{Project Information}
Company Name: RNET Technologies Inc.\\
Project Title: SASI: Smart Algorithm Selection through Inference\\
Principal Investigator: Dr. Ben O'Neill \\
Topic Number/subtopic letter: 7b (Technologies for Extreme Scale Computing (Software))

\section*{Problem Statement}
%{\em Statement of the problem or situation that is being addressed.
%  Describe the problem or situation being addressed -- be sure that
%  the Department of Energy and National interest in the problem is
%  clear.  (Typically one to three sentences).  }
While much work has taken place in optimizing the specific numerical algorithms for specific problems and architectures, there is no simple governing theory for determining the best algorithm and configuration for a given situation. Rather, the optimal method is, in practice, determined by experimentation and numerical folklore.  As such, it is desirable to have toolkit-enabled functionality that automatically chooses the algorithm that is both appropriate for the computational problem at hand \emph{and} the computer architecture that will be used. 

\section*{General Statement}
%% {\em General statement of how this problem is being addressed.  This
%%   is the overall objective of the combined Phase I and Phase II
%%   projects.  How is this problem being addressed? -- i.e., What is the
%%   overall approach of the combined Phase I/Phase II project?
%%   (Typically one to two sentences).  }
The SASI framework is a simple, robust set of tools designed to guide users through the process of building a smart algorithm
selection model for generic numerical algorithms. Through SASI, users will be able to develop tools that adapt to the computational model and 
the chosen architecture, thereby reducing the overall time to solution for numerical simulations. 

\section*{Phase I Workplan}

The goal of the Phase I effort will be to demonstrate the performance tuning capabilities of the SASI framework. To that end, the Phase I effort will include the development of the core SASI framework, the development of several novel model verification tools and an investigation into efficient data set re-use across architectures and algorithms. A demonstration whereby we use the developed framework to produce a SASI model for graph algorithms will also be provided. 

\section*{Commercial Applications and Other Benefits}
%% {\em Summarize the future applications and/or public benefits if the
%%   project is carried over into Phase II and beyond.  Do not repeat
%%   information already provided above.}
SASI automates the process of fine tuning simulations each time the model or 
architecture changes, allowing developers to focus on creating high fidelity, high 
accuracy and high impact numerical simulations capable of fully utilizing the nations 
computational resources. SASI has applications across the broad spectrum on numerical simulation
including in nuclear engineering, oil and gas, and computational fluid dynamics. 

\section*{Key Words}
%{\em Provide listing of key words that describe this effort.} 
Machine learning, numerical simulation, smart algorithm selection

\section*{Summary for Members of Congress}
%% {\em (Layman's Terms, Two Sentences Maximum, 50 words).  The
%%   Department notifies members of Congress of awards in their
%%   districts.  Therefore, please provide, in clear and concise layman's
%%   terms, a very brief summary of the project, suitable for a possible
%%   press release from a Congressional office.} 
Performance tuning of numerical simulations based on changes in the problem being solved or the computer 
being used is a tedious and time consuming task. SASI uses machine learning to remove this burden, allowing 
developers to focus on creating game changing, high fidelity simulations. 

