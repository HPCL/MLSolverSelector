\section{ Executive Summary } 
This report provides a report on DOE SBIR/STTR Phase II project: 
“Automated Solver Selection for Nuclear Engineering Simulations” with award 
number DE-SC0013869. The original project period was August 1st 2016 - 31st July 2018.
The project was granted a 12 month no cost time extension for the period August 1 2018 - July 31st
2019. As such, this document serves as a semi-annual interim report for the six month period of December 
1st 2017 - May 31st 2018. 

The project aims to address various challenges in increasing the adoption of NEAMS tools by 
novice users by enabling automatic selection of appropriate solvers for the 
given problem. Although this requires the analysis of the matrix A in the linear 
system Ax=b, there are many matrix-free methods and preconditioners which could 
benefit from this approach as well, which the framework will also support. Of 
further interest are optimizations in quantities such as the energy consumed by 
the operations and the resiliency of the solver to soft errors.

The Nuclear Energy Advanced Modeling and Simulation (NEAMS) program by DOE is
developing predictive models for the advanced nuclear reactor and fuel cycle 
systems using leading edge computational methods and high performance computing 
technologies. The SolverSelector will enable automatic selection of the optimal 
solver for the characteristics of the linear system being solved as well as 
considering the compute hardware present. 

The major accomplishment of this reporting period has been on the development of feature
extraction techniques for matrix-free methods. Feature extraction for Matrix-free methods is difficult because one does not have direct access to the matrix elements. The methods developed 
in this reporting period produce low-cost estimates of the matrix features that can be used to 
train machine learning algorithms and predict the optimal linear solver with a high level of accuracy. 

In addition to this, the project team has also continued its efforts towards developing the SolverSelector API
and in obtaining robust real-world training data sets from Proteus line of tools. 

